\documentclass[a4paper, twocolumn, 10pt]{jarticle}
\usepackage[dvipdfmx]{graphicx}
\usepackage{float}
\usepackage{setspace}
\usepackage[top=20mm,bottom=25mm,left=23mm,right=23mm]{geometry}
\usepackage[hang,small,bf]{caption}
\usepackage{bm}
\usepackage{url}
\columnsep=18pt
\renewcommand{\baselinestretch}{0.9}
%\usepackage[subrefformat=parens]{subcaption}


\makeatletter

\def\section{%
	\@startsection{section}{1}{\z@}%
	{.1\Cvs \@plus.1\Cdp \@minus.1\Cdp}%
	{.1\Cvs \@plus.1\Cdp}%
	{\normalfont\normalsize\bfseries}%
}

\def\subsection{%
	\@startsection{subsection}{1}{\z@}%
	{.1\Cvs \@plus.1\Cdp \@minus.1\Cdp}%
	{.1\Cvs \@plus.1\Cdp}%
	{\normalfont\normalsize\bfseries}%
}

\def\@maketitle {
	\begin{center}
		\fontsize{14pt}{0pt}\selectfont
		{\bf \@title}
	\end{center}
	\vspace{1pt}
	\begin{flushleft}
		{指導教員 篠埜 功}\hfill{林 祐哉}
	\end{flushleft}
	\vspace{10pt}
}

\makeatother

\captionsetup{compatibility=false}
\pagestyle{empty}


\begin{document}
\title{初心者が記述したC言語プログラムに含まれる\\
	   制御構造の途中へ飛び込むgoto文の除去アルゴリズムの設計と実装}

\maketitle

\thispagestyle{empty}

%%%%%%%%%%%%%%%%%%%%%%%%%%%%%%%%%%%%%%%%%%%
%% Section 序論
%%%%%%%%%%%%%%%%%%%%%%%%%%%%%%%%%%%%%%%%%%%
\section{序論}
C言語におけるgoto文は、指定されたラベル位置へジャンプする構文である。しかしその自由度ゆえに、プログラムの可読性が低下し、構造が崩れて理解や保守が困難になる場合がある。特に初心者が記述したプログラムでは、言語仕様や制御構造に対する理解が十分でないまま実装を進める場合があり、処理の流れを直感的に制御する手段としてgoto文が選択されることがある。例えば、繰り返し処理や条件分岐の途中から処理を再開したい場合、goto文は短い記述で目的を達成できるため、初心者にとっては書きやすい構文になる場合がある。一方で、このような使用は制御構造を破壊し、プログラム全体の処理の流れが把握しづらくなることが指摘されている\cite{1871428067537325312}。
この問題意識の出発点として、Dijkstraはgoto文が理解を妨げることを論じ\cite{dijkstra1968letters}、これ以降、if文やwhile文などの制御構文により制御フローを記述する考え方が広く浸透した。Knuthはgoto文の全面否定には慎重な立場を取りつつも、構造的記述の重要性を認めている\cite{10.1145/356635.356640}。また、goto文を制御構文へ変換するアルゴリズムも提案されてきた\cite{zegour1994new}。一方で、Ramshawは、goto文のうち複合文やループ、条件分岐の枝の中間部分へ飛び込む内側へのジャンプについて、ある種の構造を保つ条件下では除去が困難であることを示した\cite{10.1145/48014.48021}。初心者が記述したプログラムでは、例えばif文の途中ラベルへ飛び込み、その後に繰り返し処理へ合流するといった構造破壊的なジャンプが現れることがある。
一方で、goto文を除去しようとすると、入れ子の深さが過剰に深くなる、分岐が増えるなどにより、変換後に可読性が低下するという課題が残る。実際、goto文除去戦略を比較評価したCeccatoらは、除去アルゴリズムによっては入れ子深さが増えることを望ましくない副作用として挙げ、深い入れ子が理解や保守の妨げになり得ると述べている\cite{ceccato2008goto}。
したがって、初心者が記述したプログラムに頻出する内側へのジャンプを対象として、可読性を重視した実装可能な変換手法が必要である。本研究では、初心者が記述したC言語プログラムに現れるgoto文、特に内側へのジャンプを対象として、if文やwhile文などの制御構文への変換規則を設計し、実装する。さらに、変換前後の主観評価により、提案手法が理解のしやすさに寄与するかを検証する。

%%%%%%%%%%%%%%%%%%%%%%%%%%%%%%%%%%%%%%%%%%%
%% Section 提案手法(実装内容も統合)
%%%%%%%%%%%%%%%%%%%%%%%%%%%%%%%%%%%%%%%%%%%
\section{提案手法}
本研究で初心者とは、paizaスキルチェックにより基礎的なC言語能力が十分でない者を指すこととする。まず、初心者7名に3つの課題についてプログラムを作成してもらい、得られたC言語プログラム21本からgoto文を抽出する。この3つの課題とは、会員限定ページのログイン処理、テストの合計点と平均点の計算、RPGのアイテム使用処理である。各課題は、初心者がよく直面する典型的なプログラミング問題を想定しており、処理の流れを制御するためにgoto文が選択される場面が生じ得る。
各goto文についてジャンプ元とジャンプ先の位置関係から使用パターンを整理する。使用パターンのうち内側へのジャンプに該当するものに対して、制御構文を用いた変換規則を与える。変換規則の設計においては、可読性がどれだけ上がったか、構造として入れ子が深くなっていないかを重視する。
本研究では、内側へのジャンプを含むgoto文除去アルゴリズムを提案する。本研究における内側へのジャンプとは、goto文のジャンプ先のラベルが複合文、ループ、条件分岐の分岐先の内部に存在し、制御構造の途中に処理が飛び込んでしまうジャンプを指す。
また、同じ入れ子の深さであっても条件分岐の別の分岐先へ侵入する場合は構造破壊的であるため、内側へのジャンプとして扱う。
内側へのジャンプの検出には、goto文とラベル定義を抽出した上で、複合文を用い、ジャンプ先がより内側にある場合や同じ深さでも別のブロックに侵入する場合を内側へのジャンプとして判定する。
内側へのジャンプと判定されたgoto文に対しては、ジャンプの意図に応じて変換を行う。
条件分岐の分岐先の途中へ飛び込む型では、共通処理を条件分岐の外側へ移動し、差分のみをif文の中に残すことで条件分岐への侵入を不要にする。
ループ内部のラベルへ飛び込む型では、特定条件のときに処理を一度だけ実行したい意図が多いため、その処理をif文による条件付き実行へ簡約する。

成功処理や終了処理を複数の分岐から共通で実行したい場合、初心者は同じラベルへgoto文で飛んで処理をまとめる場合がある。しかしこの方法は、分岐の途中へ侵入する経路を作り、制御フローが追いにくくなる。そこで本手法では、成功したかどうかを表す変数を導入し、成功条件を満たしたときにその変数を更新する。最後にif文でその変数を判定し、成功処理を実行することで、ラベルへの飛び込みを不要にする。

また、点数入力のように合計点と人数を更新する処理では、入力とは別に与えられる追加点や補正値を、通常の点数と同じ更新処理で一回だけ加算したい意図から、ループ内部の加算処理ラベルへgoto文で飛び込む場合がある。しかしこの方法はループ内部の途中へ侵入する内側へのジャンプとなり、制御フローが追いにくくなる。そこで本手法では、ボーナス加算をループ開始前に一回だけ実行する形に正規化し、その後は通常の入力ループで点数を加算する。これにより、ボーナスを一回だけ加算するという意図を保ちながら、ループ内部への飛び込みを不要にできる。

提案手法はPythonで実装し、入力プログラムからgoto文とラベル定義を抽出した上で、ジャンプ元とジャンプ先の位置関係に基づき内側へのジャンプを検出する。その後、上記の変換規則を適用して変換後プログラムを生成する。生成後はgoto文が残存していないことを機械的に確認し、さらにgccでコンパイル可能であることを確認した。変換後のプログラムは可読性を重視し、入れ子の深さや分岐が過剰にならないように調整した。なお、本研究で抽出した内側へのジャンプを含むプログラムは7本であり、その内訳は、ループ内部のラベルへ飛び込む型が4本、条件分岐の別枝にあるラベルへ侵入する型が3本であった。以降の評価では、これら7本を対象として、提案手法による変換前後の可読性を比較した。

%%%%%%%%%%%%%%%%%%%%%%%%%%%%%%%%%%%%%%%%%%%
%% Section 実験
%%%%%%%%%%%%%%%%%%%%%%%%%%%%%%%%%%%%%%%%%%%
\section{実験}
本研究で提案したgoto文除去手法を評価するため、主観的評価を行う。まず、初心者に近い大学生7名に対し、3つの課題についてC言語プログラムを作成してもらい、得られた21本のプログラムからgoto文を抽出した。さらに、ジャンプ元とジャンプ先の位置関係に基づき、内側へのジャンプに該当するgoto文を含むプログラムを抽出したところ、該当するプログラムは7本であった。以降の主観評価は、この7本を対象として実施した。
次に、paizaスキルチェックにより基礎的なC言語能力が十分な、初心者に近い大学生7名とは異なる大学生7名を評価者として、変換前のプログラムと変換後のプログラムを提示し、可読性および理解のしやすさが向上したかをアンケートにより5段階評価してもらった。評価は5段階尺度で行い、値が大きいほど可読性および理解容易性が高いと解釈した。具体的には、評価1を「処理の流れがほとんど追えず、理解に強い負担がある」、評価3を「時間をかければ追えるが、制御フローの把握に迷いが生じる」、評価5を「上から順に自然に追え、意図した処理の流れが明確である」とし、評価2および評価4はそれぞれの中間とした。評価者には、処理の追いやすさ、条件分岐や繰り返し構造の把握しやすさ、理解に要する負担の観点から総合的に判断するよう指示した。
集計にあたっては、各プログラムについて評価者7名の評点の中央値をそのプログラムの代表値とし、7本分の代表値の分布を集計した。中央値を用いることで、個々の評価者のばらつきや外れ値の影響を抑え、プログラム単位の傾向を把握しやすくした。
アンケート結果は以下の通りである。変換前の7本のプログラムについては、代表値が評価2のものが4本、評価3のものが3本であった。一方、変換後の7本のプログラムについては、代表値が評価4のものが3本、評価5のものが4本であった。
以上より、内側へのジャンプを含むプログラムに対して提案手法を適用することで、可読性および理解のしやすさが向上する傾向が確認された。この傾向は、goto文により非局所的であった制御フローが、if文やwhile文などの構造化制御構文により上から順に追える形へ再構成されたためと考えられる。特に、条件分岐の枝の途中への飛び込みは、入口が複数存在して読み手が迷いやすいが、共通処理の外部移動や状態変数の導入により、意図が明示的になったと考えられる。また、ループ内への飛び込み型では、条件付き実行への簡約により処理の入口が一つに整理され、読者がどこから処理が始まるかを見失いにくくなったと考えられる。加算処理への飛び込み型でも、ボーナス加算を正規化してループ外に出すことで、ループの意味が明確になり、理解負担が軽減されたと考えられる。

\begin{table}[ht]
\centering
\caption{アンケート結果(対象7本, 各プログラムの評価者\\
		 7名の中央値を代表値として集計,単位:本)}
\begin{tabular}{|c|c|c|}
\hline
評価 & 変換前(本) & 変換後(本) \\
\hline
評価1 & 0本 & 0本 \\
評価2 & 4本 & 0本 \\
評価3 & 3本 & 0本 \\
評価4 & 0本 & 3本 \\
評価5 & 0本 & 4本 \\
\hline
\end{tabular}
\end{table}


%%%%%%%%%%%%%%%%%%%%%%%%%%%%%%%%%%%%%%%%%%%
%% Section まとめと今後の課題
%%%%%%%%%%%%%%%%%%%%%%%%%%%%%%%%%%%%%%%%%%%
\section{まとめと今後の課題}
提案手法に基づく変換器を実装し、初心者7名が記述したプログラム21本に適用した。その中で、内側へのジャンプを含む7本に対して提案手法を適用し、goto文をif文やwhile文などの制御構文へ変換できることを確認した。また、変換前後のプログラムを一定のプログラミング能力を持つ評価者に提示して主観評価を行い、可読性および理解のしやすさが向上する傾向がアンケート結果から明らかとなった。
さらに、変換前後の代表値は順序尺度であり、かつ同一プログラムに対する対応のある比較であるため、Wilcoxonの符号付順位検定を行った。その結果、両側検定でp=0.0156であり、変換後の評価は統計的に有意に高かった。
ただし、本手法は初心者のプログラムに頻出する内側へのジャンプの代表的パターンを対象としており、より複雑な制御フローを含むgoto文への適用範囲の拡張は今後の課題である。
今後の課題としては、より多様なgoto文の使用パターンに対応できるように、変換アルゴリズムの範囲を拡張することが挙げられる。また、入れ子の深さの抑制やアルゴリズムの単純化を行い、変換後のプログラムの品質を高めることが考えられる。さらに、goto文が複雑な場合の特殊なケースに対応するための追加的な変換規則を設計し、より多くのプログラムに対応できるようにすることが今後の研究の方向性となる。

\bibliographystyle{junsrt}
{\footnotesize \bibliography{main.bib}}
\end{document}