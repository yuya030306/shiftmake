\documentclass[a4paper, twocolumn, 10pt]{jarticle}
\usepackage[dvipdfmx]{graphicx}
\usepackage{float}
\usepackage{setspace}
\usepackage[top=20mm,bottom=20mm,left=23mm,right=23mm]{geometry}
\usepackage[hang,small,bf]{caption}
\usepackage{bm}
\usepackage{url}
%\usepackage[subrefformat=parens]{subcaption}


\makeatletter

\def\section{%
	\@startsection{section}{1}{\z@}%
	{.1\Cvs \@plus.1\Cdp \@minus.1\Cdp}%
	{.1\Cvs \@plus.1\Cdp}%
	{\normalfont\normalsize\bfseries}%
}

\def\subsection{%
	\@startsection{subsection}{1}{\z@}%
	{.1\Cvs \@plus.1\Cdp \@minus.1\Cdp}%
	{.1\Cvs \@plus.1\Cdp}%
	{\normalfont\normalsize\bfseries}%
}

\def\@maketitle {
	\begin{center}
		\fontsize{14pt}{0pt}\selectfont
		{\bf \@title}
	\end{center}
	\vspace{1pt}
	\begin{flushleft}
		{指導教員 〇〇 〇〇}\hfill{芝浦  太郎}
	\end{flushleft}
	\vspace{10pt}
}

\makeatother

\captionsetup{compatibility=false}
\pagestyle{empty}


\begin{document}

\title{論文題目}

\maketitle

\thispagestyle{empty}

%%%%%%%%%%%%%%%%%%%%%%%%%%%%%%%%%%%%%%%%%%%
%% Section 序論
%%%%%%%%%%%%%%%%%%%%%%%%%%%%%%%%%%%%%%%%%%%
\section{序論(章タイトルは,各自内容に合わせてつけること)}
このファイルは,芝浦工業大学工学部情報工学科卒研1および卒研2の概要書のテンプレートである.卒研1および卒研2の概要書はこのテンプレートを利用して作成される必要がある.

用紙サイズはA4とし,本文は2カラムとする.用紙の余白は上20mm, 下20mm, 左23mm, 右23mmとする.また,論文題目の下,および,著者氏名の下には1行程度の空白を空ける.タイトルや本文などのフォントは以下の通りとする.
\begin{itemize}
  \setlength{\itemsep}{0mm}
  \item 論文題目: ゴシック体,ボールド,14ポイント
  \item 章・節の見出し: ゴシック体,10ポイント
  \item 指導教員・著者指名: 明朝体,10ポイント
  \item 本文: 明朝体,10ポイント
  \item 図表のキャプション: 明朝体, 9ポイント
  \item 参考文献: 明朝体, 9ポイント
\end{itemize}
このテンプレートでは上述の書式が既に設定されているので,これをそのまま利用すれば問題ない.


%%%%%%%%%%%%%%%%%%%%%%%%%%%%%%%%%%%%%%%%%%%
%% Section 研究目的
%%%%%%%%%%%%%%%%%%%%%%%%%%%%%%%%%%%%%%%%%%%
\section{研究目的}
論文では,必要に応じて図や表を掲載し本文より参照すること.例えば,図の参照は『図\ref{fig_nn}に3層のニューラルネットワークを示す』,表の参照は『表\ref{table_a}に手法Aおよび手法Bの正答率を示す』などとする.図のキャプションは図の下に,表のキャプションは表の上に書く.

\begin{figure}[h]
	\centering
	\includegraphics[keepaspectratio, width=60mm]{img/sample.png}
	\caption{提案法に用いた3層のニューラルネットワーク.キャプションにはこの図の説明を書く.}
	\label{fig_nn}
\end{figure}

\begin{table}[h]
  \caption{手法Aおよび手法Bの正解率と平均計算時間.}
  \label{table_a}
  \centering
  \begin{tabular}{lcr}
    \hline
    手法   & 正解率[\%]  &  計算時間[ms]  \\
    \hline \hline
    手法A  & 92.3  & 512 \\
    手法B  & 87.4  & 32  \\
    \hline
  \end{tabular}
\end{table}


%%%%%%%%%%%%%%%%%%%%%%%%%%%%%%%%%%%%%%%%%%%
%% Section 
%%%%%%%%%%%%%%%%%%%%%%%%%%%%%%%%%%%%%%%%%%%

\section{参考文献の引用方法}

参考文献は,『井尻らは,X線CTとデジタルカメラを用いた3次元モデリング法を提案した\cite{Ijiri18}.』のように引用する.参考文献リストは,『著者1, 著者2,...,著者N. タイトル. 論文誌or学会名, 巻, 号, ページ, 発表年. 』の形式とする.
文献によっては,巻・号・ページがないものもある.著者が多い場合には,『〇〇他』や『AAA et. al.』として省略しても良い.
変化する可能性があるWebページの引用はあまり推奨されないが,Webページを引用する必要がある場合は末尾に参照日時を記入すること『著者. ページタイトル. ページURL(2021年7月31日参照) .』参考文献リストについて,ref.bibファイルに引用したい文献情報を記載しておくとLatexが自動で整形してくれるので活用すると良い.


%%%%%%%%%%%%%%%%%%%%%%%%%%%%%%%%%%%%%%%%%%%
%% Section Latexのコンパイル方法
%%%%%%%%%%%%%%%%%%%%%%%%%%%%%%%%%%%%%%%%%%%
\section{Latexファイルのコンパイル方法}

\subsection{ローカルにLatex環境を構築する場合}
各自好みの環境を使ってコンパイルするとよい.例として,TeXLiveを利用す場合は以下の手順でpdfを構築できる.
(1) TexLive\cite{TexLive}をインストール.
(2) フォルダ内のresume.texをTexLiveと同梱されているTexWorksで開く.
(3) 左上で『pBibTex』を指定しコンパイルを実行.
(4) 左上で『pLatex』を指定しコンパイルを実行(参考文献が?となる場合はこれを複数回実行).


\subsection{Overleafを利用する場合}
Overleafを利用する場合,以下の手順でpdfを作成できる.
(1)フォルダ内のresume.tex, ref.bib, latexmkrc, img/sample.pngをoverleafのプロジェクトにコピー.
(2)OVerleafのメニューより,コンパイラを『LaTeX』に,TexLive versionを2020に変更.
(3)リコンパイルボタンを押す.
もしエラーが起こる場合,(*)右側の画面からキャッシュファイルを削除する,(*)コンパイラを違うものに変更してコンパイルしてから再度LaTeXでコンパイル,(*)フォルダをつくりその中でresume.texをコンパイル,などを試すとうまくいく場合がある.


\section{まとめと展望}
最後に,概要書・卒業論文執筆前に『理科系の作文技術\cite{Kinoshita81}』および『数学文章作法 基礎編\cite{Yuki13}』を読んでおくことを勧める.
多くの学生にとって,卒業論文執筆は非常に時間のかかるタスクとなると思われる.ぜひ,余裕を持って進めてほしい.


\bibliographystyle{junsrt}
{\footnotesize \bibliography{ref.bib}}

\end{document}
