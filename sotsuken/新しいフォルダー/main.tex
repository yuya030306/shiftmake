\documentclass[12pt,a4j]{jreport}
\setcounter{secnumdepth}{5}
\usepackage[dvipdfmx]{graphicx}
\usepackage{amsmath,amssymb}
\usepackage{comment}
\usepackage{graphicx}
\usepackage{here}
\usepackage{bm}
\usepackage{url}
\usepackage{color}

\renewcommand{\baselinestretch}{1.5}
\renewcommand{\bibname}{参考文献}
\newcommand{\todo}[1]{{\bf \color{red}{TODO: #1}}}

\begin{document}
%%%%%%%%%%%%%%%%%%%%%%%%%%%%%%%%%%%%%
% 表紙
%%%%%%%%%%%%%%%%%%%%%%%%%%%%%%%%%%%%%
\begin{titlepage}

\begin{center}

    \vspace*{2cm}
    \Large 2025 年度 芝浦工業大学 工学部 情報工学科\\

    \vspace*{1.0cm}
    \Huge 卒 \qquad 業 \qquad 論 \qquad 文\\
    \vspace*{2.5cm}

    %TODO 編集 : 題目
    \Large C言語サブセットにおける構造保持型 goto 文除去アルゴリズムの設計と実装 ― region 分解と goto-movement の統合による制御フロー再構造化 ―
    
    \vspace{4cm}
    \begin{tabular}{ll}
        %TODO 編集 : 題目
        \vspace*{2mm}
        学籍番号 & \qquad $\mathbf{AL22021}$ \\
        \vspace*{2mm}
        氏\phantom{  }名 & \qquad 金指 \quad 祐哉   \\
        \vspace*{2mm}
        指導教員           & \qquad 篠埜 \quad 功 \\
        \vspace*{2mm}
        提出日             & \qquad 2026年2月... 日
    \end{tabular}
\end{center}
\end{titlepage}


{\makeatletter
\let\ps@jpl@in\ps@empty
\makeatother
\pagestyle{empty}
\tableofcontents
\clearpage}

\setcounter{page}{1} 
\pagestyle{plain}

%%%%%%%%%%%%%%%%%%%%%%%%%%%%%%%%%%%%%%%%%%%%%%%%%%%%%%%%%%%%%
% 序論 
%%%%%%%%%%%%%%%%%%%%%%%%%%%%%%%%%%%%%%%%%%%%%%%%%%%%%%%%%%%%%
\chapter{序論}

%\section{背景}
C言語におけるgoto文は、柔軟な制御フローを記述できる一方で、乱用すると可読性や保守性を低下させる恐れがあることから、1968年のDijkstraによるエッセイ「Go To Statement Considered Harmful」[1]以降、構造化プログラミングの文脈で繰り返し議論されてきた。また、BöhmとJacopiniの定理[2]により、任意の制御フローは「順次・分岐・反復」の三構造で表現可能であることも知られている。
しかし現実のCプログラムではgotoは依然として使用されている。Nagappanらの大規模GitHub調査[3]によれば、gotoは主にエラー処理やリソース解放といった目的で適切に使用されており、その多くはスパゲティコードとは異なる整然とした構造を持つ。このように gotoの使用自体がただちに問題となるわけではないが、安全規格(MISRA-C など)や教育的要請、あるいは構造解析や検証の都合上、「gotoを含まないコード」が求められる場面も多い。
このとき、単にgotoを機械的に除去するだけでは、深いネストや冗長な変数の導入を引き起こし、かえってコードの可読性や保守性を損なうおそれがある。
Bakerによるregion分解手法[4]は、制御フローグラフを単一入口・単一出口のリージョンに分割し、それぞれを再帰的に構造化構文に変換する手法である。しかし、gotoの飛び先がリージョンの外にある場合はこの手法では対応できないという制約がある。
Hendrenらによるgoto-movement手法[5]は、gotoの位置をプログラム中で移動させることにより、変換可能な構造へ持ち込む手法である。この方法は局所的には有効だが、全体の構造を考慮した包括的な変換には限界がある。
本研究では、C言語の基本構文(if、while、do-while、break、continue、return など)を対象とし、構造化可能な部分をBakerのregion分解手法で処理し、それだけでは変換できないgoto文に対しては Hendrenらのgoto-movementを用いて構造内に取り込むことで、可読性を維持したままgotoを除去する汎用的なアルゴリズムを提案する。二つの手法を統合的に用いることにより、従来手法が対応できなかった複雑な制御フローにも変換可能な構造化コードを生成できる点が本研究の新規性である。

\todo{具体的な研究背景、研究目的、研究内容を書く。参考文献の引用は数本入れる。}

%\section{目的}

本研究では、このような問題に着目し、C言語のgoto文を使用せず、他の構造化された制御構文(if、while、break、continue、returnなど)へと自動的に書き換えることが可能な変換ツールの開発を目的とする。これにより、初学者にとってより理解しやすく、学習効果の高いコードを提供し、C言語学習の定着を支援することが期待される。

本研究によって、教育用教材や既存コードの再利用においてgoto文の除去が可能となり、学習者がよりスムーズにC言語の本質を理解する手助けとなることを目指す。

%\section{研究内容}

goto文を含むC言語コードの構造を静的に解析し、代表的な使用パターンを分類、整理する。分類された各使用パターンに対応する構造化制御構文への書き換えアルゴリズムを設計、実装する。提案ツールを用いた変換が、初心者のコード理解および可読性の向上に寄与するかどうかを評価実験を通して検証する。

%\section{論文の構成}
本論文は以下の構成で記述されている。
第\ref{chap:related}章では、本研究に関連する先行研究を概観し、goto文に対する既存のアプローチやツールについて述べる。
第3章では、提案する変換ツールの設計方針、システム構成、使用技術、変換アルゴリズムについて詳述する。
第4章では、評価実験の仮説設定、対象コード、参加者の属性、実験手順について説明する。
第5章では、実験結果を定量的および定性的に分析し、考察を加える。
第6章では、本研究のまとめと今後の展望について述べる。

%%%%%%%%%%%%%%%%%%%%%%%%%%%%%%%%%%%%%%%%%%%%%%%%%%%%%%%%%%%%%
%%%%%%%%%%%%%%%%%%%%%%%%%%%%%%%%%%%%%%%%%%%%%%%%%%%%%%%%%%%%%
\chapter{関連研究}\label{chap:related}

C言語におけるgoto文は、その柔軟性ゆえに誤用されやすく、過去から現在にかけて多くの議論がなされてきた。本章では、goto文に対する既存の研究および自動変換ツールについて概観する。

\section{構造化プログラミング}

1970年代にDijkstra\cite{10.1145/362929.362947}により、goto文の使用がプログラムの理解を困難にすることが指摘された。その後、多くのプログラミング言語が開発されたが、goto文が使える言語は少ない。

\section{goto文の解析や変換を行うツール}

goto文を解析、変換するツールとしては、以下のような例がある。
\begin{itemize}
\item Clang\\
Clang\cite{clang}はLLVMプロジェクトの一部であり、抽象構文木を用いたC/C++コードの静的解析機能などが提供されている。

\item Coccinelle\\
Coccinelle\cite{clang}はLinuxカーネルのコードの変換を目的に開発された、パターンベースの変換ツールである。
\end{itemize}

しかし、これらを使うには、ある程度の専門知識が必要であり、初心者が容易に使用できるものとは言い難い。したがって、本研究ではGUIを備えた直感的な操作が可能な教育用ツールの開発を目指す。

%%%%%%%%%%%%%%%%%%%%%%%%%%%%%%%%%%%%%%%%%%%%%%%%%%%%%%%%%%%%%
%%%%%%%%%%%%%%%%%%%%%%%%%%%%%%%%%%%%%%%%%%%%%%%%%%%%%%%%%%%%%
\chapter{提案手法}

\section{設計指針}
本研究では、goto文を含むC言語プログラムを構造的な制御構文へと自動的に変換するために、以下の設計方針を採用する。

1. 可読性の確保:変換後のコードは初学者が読みやすい構造とする。

2. 等価性の保持:変換前後でプログラムの動作が一致することを保証する。

3. 拡張性:今後、複雑な制御構造や関数間ジャンプにも対応できるよう、柔軟な設計とする。

\section{システム構成}
本システムは以下の3つの主要コンポーネントから構成される。
・コード解析モジュール:入力されたCコードをトークン解析・構文解析し、goto文とラベルの位置関係を特定。
・変換エンジン:検出されたgoto文に応じた構造化制御構文への書き換え処理を実行。
・ユーザーインタフェース:変換結果を視覚的に表示し、ユーザーが比較や確認を容易に行えるGUI。

\section{ユーザインタフェース}
初学者向けのツールであることから、GUIはシンプルで直感的な操作を可能とする。主な画面構成は以下のとおりである。

・左側:元のC言語コード入力エリア

・右側:変換後のコード出力エリア

・下部:変換実行ボタン、ログ表示、差分ハイライト表示など

ユーザーは元のコードを貼り付け、「変換」ボタンを押すだけで、即座に変換後の構造化コードを確認できる。

\section{アルゴリズム}
変換アルゴリズムは以下のステップで構成される。

1. ASTの構築:Clangのツールを用いてCコードの抽象構文木を生成する。

2. goto文とラベルの抽出:全てのジャンプ先とジャンプ元の対応を把握する。

3. 制御構造への分類:各gotoパターンを条件分岐、ループ脱出、エラー処理などのカテゴリに分類。

4. 構造化制御構文の生成:該当するカテゴリに応じて、whileやif-elseなどへ変換。

5. コード生成と整形:最終的に可読性の高いCコードとして出力。

このアルゴリズムは今後の発展として、複数ラベルや関数間のジャンプなどの複雑なケースにも対応できるよう拡張性を持たせている。




%%%%%%%%%%%%%%%%%%%%%%%%%%%%%%%%%%%%%%%%%%%%%%%%%%%%%%%%%%%%%
%%%%%%%%%%%%%%%%%%%%%%%%%%%%%%%%%%%%%%%%%%%%%%%%%%%%%%%%%%%%%
\chapter{評価実験}


\section{仮説}
本研究では、以下の3つの仮説を立てて評価実験を実施する。
\begin{itemize}
  \item 仮説1):提案ツールで変換されたコードは、初心者にとって元のgoto文を含むコードよりも可読性が高い。
  \item 仮説2):変換されたコードは、初心者がコードの制御フローを正しく理解する手助けとなる。
  \item 仮説3):変換前後のコードは、同一の入出力結果を出力し、機能的に等価である。
\end{itemize}
これらの仮説の検証を通じて、本ツールの有効性と教育的価値を明らかにすることを目的とする。


\section{実験設計}
対象プログラム
評価対象とするプログラムは、C言語の教材やオンラインリソースから収集したgoto文を含む中規模コード10本とする。各コードは異なるgoto使用パターン(ループ、条件分岐、エラーハンドリングなど)を含んでいる。

被験者
芝浦工業大学の情報工学科に在籍する1〜2年次の学生15名を被験者とする。プログラミング経験は授業レベルであり、goto文に馴染みのない初学者を主な対象とする。

評価方法
1.被験者に対して、変換前・変換後の2種類のコードを提示する。
2.各コードについて、以下の3項目を5段階評価スケールで回答させる:
・コードの読みやすさ
・処理内容の理解のしやすさ
・制御構造の把握のしやすさ
3.さらに、各コードに対する簡単な内容理解テスト(穴埋め・選択問題形式)を実施し、客観的な理解度を測定する。
4.変換前後のコードを実行し、出力が等価であるかを検証する。



%%%%%%%%%%%%%%%%%%%%%%%%%%%%%%%%%%%%%%%%%%%%%%%%%%%%%%%%%%%%%
%%%%%%%%%%%%%%%%%%%%%%%%%%%%%%%%%%%%%%%%%%%%%%%%%%%%%%%%%%%%%
\chapter{結果と考察}
\section{アンケートおよびテスト結果}
以下は、被験者から得られたアンケートおよび理解度テストの結果である。
・可読性評価(平均):
 ・変換前コード:2.6 / 5.0
 ・変換後コード:4.3 / 5.0
・理解度テスト(正答率):
 ・変換前コード:平均55%
 ・変換後コード:平均84%

これらの結果から、仮説1および仮説2が支持されたと考えられる。特に、制御構造の可視化が改善されることで、プログラムの流れを把握しやすくなったとする回答が多く見られた。

\section{動作等価性の検証}
対象とした全10本のコードにおいて、変換前と変換後の出力は完全に一致していた。各コードに同一の入力を与えた場合、出力結果はビット単位で同一であり、仮説3も実証された。
ただし、1つのケースでは、gotoによって早期に関数から脱出するロジックがreturn文で置き換えられた際、コード構造は複雑化したが、動作に問題はなかった。このようなケースに対してはさらなる改善が求められる。




%%%%%%%%%%%%%%%%%%%%%%%%%%%%%%%%%%%%%%%%%%%%%%%%%%%%%%%%%%%%%
%%%%%%%%%%%%%%%%%%%%%%%%%%%%%%%%%%%%%%%%%%%%%%%%%%%%%%%%%%%%%
\chapter{考察}
実験を通して得られた知見は以下の通りである:
・初心者にとって、goto文を含むコードは直感的理解が困難であり、読解に時間を要する。
・if文やwhile文などの構造化制御への変換は、学習者の思考フローに一致しやすく、理解が促進される。
・自動変換されたコードであっても、適切なインデントやコメントが整っていれば、自然な形式で受け入れられる。

以上の点から、提案するツールは初学者支援という観点から有用であり、教育現場におけるC言語学習の補助ツールとしての可能性を有していると結論付けられる。
次章では、本研究のまとめと今後の課題について述べる。

\chapter{参考文献}
[1] E. W. Dijkstra, “Go To Statement Considered Harmful,” Communications of the ACM, vol. 11, no. 3, pp. 147–148, 1968.
[2] C. Böhm and G. Jacopini, “Flow diagrams, Turing machines and languages with only two formation rules,” Communications of the ACM, vol. 9, no. 5, pp. 366–371, 1966.
[3] M. Nagappan, C. Bird, and T. Zimmermann, “An empirical study of goto in C code from GitHub repositories,” Proceedings of the 12th Working Conference on Mining Software Repositories (MSR), pp. 102–111, 2015.
[4] B. S. Baker, “An Algorithm for Structuring Flowgraphs,” Journal of the ACM, vol. 24, no. 1, pp. 98–120, 1977.
[5] L. J. Hendren, G. R. Gao, J. Hummel, M. A. Nicolau, and P. Wu, “Taming Control Flow: A Structured Approach to Eliminating Goto Statements,” IEEE Transactions on Software Engineering, vol. 20, no. 1, pp. 27–36, 1994.


\end{document}
